\documentclass[10pt, a4paper]{extarticle}
\setlength{\parskip}{0.5em}
%%% Работа с русским языком
\usepackage{cmap}					% поиск в PDF
\usepackage{mathtext} 				% русские буквы в формулах
\usepackage[T2A]{fontenc}			% кодировка
\usepackage[utf8]{inputenc}			% кодировка исходного текста
\usepackage[english,russian]{babel}	% локализация и переносы
\usepackage{mathtools}   % loads »amsmath«
\usepackage{graphicx}
\usepackage{caption}
\usepackage{physics}
\usepackage{subcaption}
\usepackage{tikz}

%%% Дополнительная работа с математикой
\usepackage{amsmath,amsfonts,amssymb,amsthm,mathtools} % AMS
\usepackage{icomma} % "Умная" запятая: $0,2$ --- число, $0, 2$ --- перечисление

%% Шрифты
\usepackage{euscript}	 % Шрифт Евклид
\usepackage{mathrsfs} % Красивый матшрифт

\title{Сюжеты с кружка по эконометрике}

\usepackage{geometry}
\geometry{
	a4paper,
	left=20mm,
	top=20mm,
	right=20mm
}
\setlength{\parindent}{0cm}

\let\P\relax
\DeclareMathOperator{\P}{\mathbb{P}}
\DeclareMathOperator{\E}{\mathbb{E}}

\begin{document}
	
\maketitle

\section{Идеи распределений}

\subsection{Экспоненциальное распределение и распределение Пуассона через АПП}

Аксиомы Пуассоновского потока (АПП):
\begin{enumerate}
	\item Время непрерывно.
	\item На оси времени происходят точечные происшествия.	
	\begin{center}
	\tikzset{every picture/.style={line width=0.75pt}} %set default line width to 0.75pt        
	\begin{tikzpicture}[x=0.75pt,y=0.75pt,yscale=-1,xscale=1]
	%uncomment if require: \path (0,137); %set diagram left start at 0, and has height of 137
	
	%Straight Lines [id:da6929201120835929] 
	\draw    (100,68) -- (387.5,67.01) ;
	\draw [shift={(389.5,67)}, rotate = 539.8] [color={rgb, 255:red, 0; green, 0; blue, 0 }  ][line width=0.75]    (10.93,-3.29) .. controls (6.95,-1.4) and (3.31,-0.3) .. (0,0) .. controls (3.31,0.3) and (6.95,1.4) .. (10.93,3.29)   ;
	
	%Shape: Circle [id:dp3127741969749247] 
	\draw  [fill={rgb, 255:red, 0; green, 0; blue, 0 }  ,fill opacity=1 ] (129,67.5) .. controls (129,65.01) and (131.01,63) .. (133.5,63) .. controls (135.99,63) and (138,65.01) .. (138,67.5) .. controls (138,69.99) and (135.99,72) .. (133.5,72) .. controls (131.01,72) and (129,69.99) .. (129,67.5) -- cycle ;
	%Shape: Circle [id:dp5786911431724271] 
	\draw  [fill={rgb, 255:red, 0; green, 0; blue, 0 }  ,fill opacity=1 ] (218,67.5) .. controls (218,65.01) and (220.01,63) .. (222.5,63) .. controls (224.99,63) and (227,65.01) .. (227,67.5) .. controls (227,69.99) and (224.99,72) .. (222.5,72) .. controls (220.01,72) and (218,69.99) .. (218,67.5) -- cycle ;
	%Shape: Circle [id:dp5938248751179869] 
	\draw  [fill={rgb, 255:red, 0; green, 0; blue, 0 }  ,fill opacity=1 ] (290,67.5) .. controls (290,65.01) and (292.01,63) .. (294.5,63) .. controls (296.99,63) and (299,65.01) .. (299,67.5) .. controls (299,69.99) and (296.99,72) .. (294.5,72) .. controls (292.01,72) and (290,69.99) .. (290,67.5) -- cycle ;
	
	% Text Node
	\draw (386,83) node   {$t$};
	\end{tikzpicture}
	\end{center}
	<<Точечные>> означает, что происшествие не длится, скажем, 2 секунды, а случается <<в точке>> на оси времени.
	
	\item Количества происшествий на непересекающихся интервалах независимы.
	\begin{center}
	\tikzset{every picture/.style={line width=0.75pt}} %set default line width to 0.75pt        
	
	\begin{tikzpicture}[x=0.75pt,y=0.75pt,yscale=-1,xscale=1]
	%uncomment if require: \path (0,137); %set diagram left start at 0, and has height of 137
	
	%Straight Lines [id:da6929201120835929] 
	\draw    (14,56) -- (301.5,55.01) ;
	\draw [shift={(303.5,55)}, rotate = 539.8] [color={rgb, 255:red, 0; green, 0; blue, 0 }  ][line width=0.75]    (10.93,-3.29) .. controls (6.95,-1.4) and (3.31,-0.3) .. (0,0) .. controls (3.31,0.3) and (6.95,1.4) .. (10.93,3.29)   ;
	
	%Shape: Rectangle [id:dp30649803725484637] 
	\draw   (27,46) -- (90.5,46) -- (90.5,66) -- (27,66) -- cycle ;
	%Shape: Rectangle [id:dp7427458030588374] 
	\draw   (154.5,45) -- (264.5,45) -- (264.5,65) -- (154.5,65) -- cycle ;
	
	% Text Node
	\draw (300,71) node   {$t$};
	
	
	\end{tikzpicture}
	\end{center}
	\item Закон распределения количества происшествий стабилен во времени. Это означает, что на двух временных интервалах одинаковой длины количества происшествий распределены одинаково.
	
	\item На малом интервале времени $\Delta t$:
	\begin{itemize}
		\item вероятность двух и более происшествий мала по сравнению с $\Delta t$:
		\[
		\P(\text{2 и более происшетсвий}) = o(\Delta t).
		\]
		\item вероятность ровно одного происшествия пропорциональна $\Delta t$ с точностью до $o$-малых:
		\[
		\P(\text{ровно 1 происшествие}) = \lambda\Delta t + o(\Delta t).
		\]
	\end{itemize}

\end{enumerate}

Посмотрим, как из АПП можно вывести экспоненциальное распределение и распределение Пуассона. Рассмотрим следующий участок на оси времени:
\begin{center}
\tikzset{every picture/.style={line width=0.75pt}} %set default line width to 0.75pt        

\begin{tikzpicture}[x=0.75pt,y=0.75pt,yscale=-1,xscale=1]
%uncomment if require: \path (0,101.00001525878906); %set diagram left start at 0, and has height of 101.00001525878906

%Straight Lines [id:da6929201120835929] 
\draw    (14,56) -- (301.5,55.01) ;
\draw [shift={(303.5,55)}, rotate = 539.8] [color={rgb, 255:red, 0; green, 0; blue, 0 }  ][line width=0.75]    (10.93,-3.29) .. controls (6.95,-1.4) and (3.31,-0.3) .. (0,0) .. controls (3.31,0.3) and (6.95,1.4) .. (10.93,3.29)   ;

%Shape: Circle [id:dp4743734570710031] 
\draw  [fill={rgb, 255:red, 0; green, 0; blue, 0 }  ,fill opacity=1 ] (68.25,55.63) .. controls (68.25,53.21) and (70.21,51.25) .. (72.63,51.25) .. controls (75.04,51.25) and (77,53.21) .. (77,55.63) .. controls (77,58.04) and (75.04,60) .. (72.63,60) .. controls (70.21,60) and (68.25,58.04) .. (68.25,55.63) -- cycle ;
%Shape: Circle [id:dp8739714732845996] 
\draw  [fill={rgb, 255:red, 0; green, 0; blue, 0 }  ,fill opacity=1 ] (128.25,55.63) .. controls (128.25,53.21) and (130.21,51.25) .. (132.63,51.25) .. controls (135.04,51.25) and (137,53.21) .. (137,55.63) .. controls (137,58.04) and (135.04,60) .. (132.63,60) .. controls (130.21,60) and (128.25,58.04) .. (128.25,55.63) -- cycle ;
%Shape: Circle [id:dp5886124486572519] 
\draw  [fill={rgb, 255:red, 0; green, 0; blue, 0 }  ,fill opacity=1 ] (216.25,55.63) .. controls (216.25,53.21) and (218.21,51.25) .. (220.63,51.25) .. controls (223.04,51.25) and (225,53.21) .. (225,55.63) .. controls (225,58.04) and (223.04,60) .. (220.63,60) .. controls (218.21,60) and (216.25,58.04) .. (216.25,55.63) -- cycle ;

% Text Node
\draw (300,71) node   {$t$};
% Text Node
\draw (74,71) node   {$0$};
% Text Node
\draw (134,70) node   {$a$};
% Text Node
\draw (225,70) node   {$a\ +\ \Delta a$};


\end{tikzpicture}
\end{center}

Попробуем найти вероятность того, что за промежуток времени $[0; a]$ произойдёт ровно 0 происшествий:
\[
\P(\text{0 происшествий за } [0; a]).
\]

Для этого рассмотрим вероятность того, что ровно 0 происшествий произойдёт за промежуток времени $[0; a + \Delta a]$:
\[
\P(\text{0 происшествий за } [0; a + \Delta a]).
\]
Так как $[0; a]$ и $[a; a + \Delta a]$ -- непересекающиеся интервалы, то по аксиоме 3 количество происшествий на них независимы, а значит совместная вероятность раскладывается в произведение:
\[
\P(\text{0 происшествий за } [0; a + \Delta a]) = \P(\text{0 происшествий за } [0; a]) \times \P(\text{0 происшествий за } [a; a + \Delta a]).
\]

А $\P(\text{0 происшествий за } [a; a + \Delta a])$ можно рассчитать по аксиоме 5. На этом интервале может произойти ноль, одно или два и боле событий. Так как нас интересует первый вариант, вычтем из единицы вероятности второго и третьего вариантов:
\begin{multline*}
\P(\text{0 происшествий за } [a; a + \Delta a]) = 1 - \P(\text{1 происшествие за } [a; a + \Delta a]) -\\- \P(\text{2 и более происшествий за } [a; a + \Delta a]).
\end{multline*}

А эти вероятности -- ровно то, что стоит в аксиоме 5, только в нашем случае интервал $\Delta t$ -- это $\Delta a$. Таким образом, получаем:
\begin{multline*}
\P(\text{0 происшествий за } [a; a + \Delta a]) = 1 - \P(\text{1 происшествие за } [a; a + \Delta a]) -\\- \P(\text{2 и более происшествий за } [a; a + \Delta a]) =\\= 1 - \lambda \Delta a - o(\Delta a) - o(\Delta a) = 1 - \lambda \Delta a - o(\Delta a) \text{ (по свойствам $o$-малых)}.
\end{multline*}

Вернёмся к начальному выражению:
\[
 \P(\text{0 происшествий за } [0; a + \Delta a]) = \P(\text{0 происшествий за } [0; a]) \times (1 - \lambda \Delta a - o(\Delta a)).
\]
 
Раскроем скобки:
\begin{multline*}
\P(\text{0 происшествий за } [0; a + \Delta a]) - \P(\text{0 происшествий за } [0; a]) = \P(\text{0 происшествий за } [0; a])(-\lambda \Delta a) + o(\Delta a),
\end{multline*}
так как $\P(\text{0 происшествий за } [0; a]) \times o(\Delta a) = o(\Delta a)$ (вероятность лежит в пределах от нуля до единицы, то есть мала).

Поделим обе части на $\Delta a$:
\begin{multline*}
\dfrac{\P(\text{0 происшествий за } [0; a + \Delta a]) - \P(\text{0 происшествий за } [0; a])}{\Delta a} = \P(\text{0 происшествий за } [0; a])(-\lambda) + \dfrac{o(\Delta a)}{\Delta a}.
\end{multline*}

А теперь возьмём предел правой и левой части при $\Delta a \rightarrow 0$. Заметим, что слева стоит ни что иное как производная $\P(\text{0 происшествий за } [0; a])$ (по определению производной), если мы рассматриваем эту вероятность как функцию. Также по определению:
\[
\lim\limits_{\Delta a \rightarrow 0} \dfrac{o(\Delta a)}{\Delta a} = 0.
\]

Обозначим вероятность $\P(\text{0 происшествий за } [0; a])$ как $z(a)$, то есть явно скажем, что это некоторая функция:
\[
\P(\text{0 происшествий за } [0; a]) \equiv z(a).
\]

Тогда выражение выше можно записать как:
\[
z'_{a} = -\lambda z(a) + 0.
\]

Общее решение этого дифференциального уравнения:
\[
z(a) = C e^{-\lambda a}.
\]

Заметим, что $z(a)$ уже похожа на функции распределения экспоненциального распределения и распределения Пуассона. Восстановим константу $C$. Логично предположить, что $z(0) = 1$, то есть вероятность того, что за промежуток времени $[0; 0]$ произойдёт 0 происшествий равна единице. Отсюда:
\[
1 = Ce^0 \Rightarrow C = 1.
\]

Таким образом: 
\[
z(a) = e^{-\lambda a}.
\]

Что такое $z(a)$? По нашей записи это вероятность того, что в промежуток времени $[0; a]$ произойдёт 0 происшествий. 

Распределение Пуассона как раз моделирует случайную величину, которая показывает число событий, произошедших за фиксированный промежуток времени. Предположим, что $a = 1$, то есть $z(a) = e^{-\lambda}$ ( промежуток времени фиксирован от 0 до 1). Таким образом, $z(a)$ -- это вероятность того, что случайная величина, имеющая Пуассоновское распределение, примет значение 0 на промежутке времени $[0; 1]$. Покажем это:
\[
X \sim pois(\lambda),
\]
\[
\P(X = 0) = \dfrac{e^{-\lambda} \times \lambda^0}{0!} = e^{-\lambda}.
\]

Заметим, что $z(a)$ можно интерпретировать по-другому: если это вероятность того, что за промежуток времени $[0; a]$ произойдёт 0 происшествий, то это же вероятность того, что первое происшествие произойдёт после точки $a$ на временной оси. Пусть случайная величина $Y$ показывает время до первого происшествия. Тогда:
\[
z(a) = \P(Y \ge a).
\]
Перепишем данное выражение в терминах функции распределения:
\[
P(Y \le a) = 1 - z(a) = 1 - e^{-\lambda a}.
\]
Получили функцию распределения экспоненциального распределения! Экспоненциальное распределение моделирует время между двумя происшествиями. У нас получилось, что $Y \sim exp(\lambda)$ и $Y$ -- время до первого происшествия, то есть экспоненциальное распределение также показывает время от начала отсчёта до первого происшествия. Ввиду свойства отсутствия памяти, два этих определения эквивалентны (так как после наступления нового происшествия, точка отсчёта смещается в точку этого происшествия).

Покажем, что распределение Пуассона можно вывести для любого числа происшествий. Например, проверим, что наши рассуждения верны и для $\P(\text{1 происшествие за } [0; a])$. Будем рассматривать фиксированный промежуток времени $[0; 1]$ (то есть при $a = 1$). Ожидаемый результат:
\[
X \sim pois(\lambda).
\]
\[
\P(X=1) = e^{-\lambda}\lambda.
\]

Для краткости обозначим искомую вероятность за $u(a)$:
\[
\P(\text{1 происшествие за } [0; a]) \equiv u(a).
\]

Применим ту же схему. Вероятность того, что произошло ровно 1 происшествие за промежуток времени $[0; a + \Delta a]$ раскладывается в сумму вероятностей, что ровно 1 происшествие произошло либо за $[0; a]$, либо за $[a; a + \Delta a]$. Первая вероятность -- это произведение вероятностей, что за $[0; a]$ произошло ровно одно происшествие, а за $[a + \Delta a]$ произошло ноль происшествий. Вторая вероятность -- это произведение вероятностей, что за $[0; a]$ произошло ноль происшествий, а за $[a + \Delta a]$ произошло ровно одно происшествие. Запишем эти рассуждения в наших обозначениях:
\[
u(a + \Delta a) = u(a)(1 - \lambda \Delta a - o(\Delta a)) + z(a)(\lambda \Delta a + o(\Delta a)).
\]

Снова раскроем скобки и поделим обе части на $\Delta a$:
\[
\dfrac{u(a + \Delta a) - u(a)}{\Delta a} = \lambda z(a) - \lambda u(a) + \dfrac{o(\Delta a)}{\Delta a}.
\]

В пределе при $\Delta a \rightarrow 0$ получаем:
\[
u'_a = \lambda e^{-\lambda a} - \lambda u(a).
\]
Условие $u(0) = 0$ опять же выполняется. Если решить данное дифференциальное уравнение, получим:
\[
u(a) = \dfrac{e^{-\lambda a} \lambda a}{1!}.
\]
При $a = 1$ получаем:
\[
u(a) = e^{-\lambda} \lambda.
\]

Получили, что ожидали. Можно проверить результаты и для большего числа происшествий. 

Важный факт, который получаем из двоякой интерпретации $z(a)$: если предполагаем, что время между двумя происшествиями распределено экспоненциально, то их количество распределено по Пуассону. И наоборот, если считаем, что количество происшествий распределено по Пуассону, то время между двумя происшествиями распределено экспоненциально.

 
\end{document}